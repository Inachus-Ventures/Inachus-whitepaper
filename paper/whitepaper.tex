%%%%%%%%%%%%%%%%%%%%%%%%%%%%%%%%%%%%%%%%%
% Journal Article
% LaTeX Template
% Version 1.4 (15/5/16)
%
% This template has been downloaded from:
% http://www.LaTeXTemplates.com
%
% Original author:
% Frits Wenneker (http://www.howtotex.com) with extensive modifications by
% Vel (vel@LaTeXTemplates.com)
%
% License:
% CC BY-NC-SA 3.0 (http://creativecommons.org/licenses/by-nc-sa/3.0/)
%
%%%%%%%%%%%%%%%%%%%%%%%%%%%%%%%%%%%%%%%%%

%----------------------------------------------------------------------------------------
%	PACKAGES AND OTHER DOCUMENT CONFIGURATIONS
%----------------------------------------------------------------------------------------

\documentclass[twoside,twocolumn]{article}

\usepackage{blindtext} % Package to generate dummy text throughout this template 

\usepackage[sc]{mathpazo} % Use the Palatino font
\usepackage[T1]{fontenc} % Use 8-bit encoding that has 256 glyphs
\linespread{1.05} % Line spacing - Palatino needs more space between lines
\usepackage{microtype} % Slightly tweak font spacing for aesthetics

\usepackage[english]{babel} % Language hyphenation and typographical rules

\usepackage[hmarginratio=1:1,top=32mm,columnsep=20pt]{geometry} % Document margins
\usepackage[hang, small,labelfont=bf,up,textfont=it,up]{caption} % Custom captions under/above floats in tables or figures
\usepackage{booktabs} % Horizontal rules in tables

\usepackage{lettrine} % The lettrine is the first enlarged letter at the beginning of the text

\usepackage{enumitem} % Customized lists
\setlist[itemize]{noitemsep} % Make itemize lists more compact

\usepackage{abstract} % Allows abstract customization
\renewcommand{\abstractnamefont}{\normalfont\bfseries} % Set the "Abstract" text to bold
\renewcommand{\abstracttextfont}{\normalfont\small\itshape} % Set the abstract itself to small italic text

\usepackage{titlesec} % Allows customization of titles
\renewcommand\thesection{\Roman{section}} % Roman numerals for the sections
\renewcommand\thesubsection{\roman{subsection}} % roman numerals for subsections
\titleformat{\section}[block]{\large\scshape\centering}{\thesection.}{1em}{} % Change the look of the section titles
\titleformat{\subsection}[block]{\large}{\thesubsection.}{1em}{} % Change the look of the section titles

\usepackage{fancyhdr} % Headers and footers
\pagestyle{fancy} % All pages have headers and footers
\fancyhead{} % Blank out the default header
\fancyfoot{} % Blank out the default footer
\fancyhead[C]{Running title $\bullet$ May 2016 $\bullet$ Vol. XXI, No. 1} % Custom header text
\fancyfoot[RO,LE]{\thepage} % Custom footer text

\usepackage{titling} % Customizing the title section

\usepackage{hyperref} % For hyperlinks in the PDF

%----------------------------------------------------------------------------------------
%	TITLE SECTION
%----------------------------------------------------------------------------------------

\setlength{\droptitle}{-4\baselineskip} % Move the title up

\pretitle{\begin{center}\Huge\bfseries} % Article title formatting
    \posttitle{\end{center}} % Article title closing formatting
\title{The Inachus.io Protocol} % Article title
\author{
  \textsc{Daniel Kaminski de Souza}\thanks{Inachus Ventures Chief Technology Officer} \\[1ex] % Your name
  \normalsize Inachus Ventures \\ % Your institution
  \normalsize \href{mailto:daniel@inachus.ventures}{daniel@inachus.ventures} % Your email address
  \and % Uncomment if 2 authors are required, duplicate these 4 lines if more
  \textsc{Amish Patel}\thanks{Inachus Ventures Chief Executive Officer} \\[1ex] % Your name
  \normalsize Inachus Ventures \\ % Your institution
  \normalsize \href{mailto:amish@inachus.ventures}{amish@inachus.ventures} % Your email address}
}
\date{\today} % Leave empty to omit a date
\renewcommand{\maketitlehookd}{%
  \begin{abstract}
    \noindent This paper presents the technology that powers the Inachus.io Protocol % Dummy abstract text - replace \blindtext with your abstract text
  \end{abstract}
}

%----------------------------------------------------------------------------------------

\begin{document}

% Print the title
\maketitle

%----------------------------------------------------------------------------------------
%	ARTICLE CONTENTS
%----------------------------------------------------------------------------------------

\section{Introduction}

\lettrine[nindent=0em,lines=3]{T} he achievement of efficient sustainable organisational growth can be considered a multi-objective optimisation problem where organisations look for minimising resources cost and maximising growth. The United Nations together with dozens of nations classified development across 17 goals (SDG goals)~\cite{the_17_sdg_goals} in relevant areas of interest to ethically support life. The SDG goals have enough independence from each other that they can be classified as ortogonal 17 dimensions. Growth in one of these dimensions may come at the expense of shrinkage in another, that is why it is important for organisations to measure and track progress in every dimension as they are equaly valuable to society. With this classification, optimisation tools from the artificial intelligence field of computer science may be applied to size resources allocation to reach most efficient overall sustainable growth when satisfactory models to estimate resulting growth in each dimension are provided. Positive social change from the proposed SDG dimensions perspective can be incentivised by rewarding virtually cost free tokens to responsible impactful entities accordingly to impact produced. The quality of the metrics used to evaluate impact in each SDG dimension allows circular economy experts to come up, for example, to the conclusion that each pound invested in buying water quality tokens results in 4 pounds savings in health care. With this important information, entities interested on the net 400\% profitability can start buying water tokens from the market resulting in water token appreciation which incentivises entities to either invest in water quality tokens or to effecively produce impact that increases the water quality in exchange for water token rewards. Supporting this vision, we are creating a family of tokens designed to promote circular economy advances in each one of the 17 SDG dimensions. Organisations that are effecively able to measure concrete impact in a given dimension will be able to apply for a minting authority.

% %------------------------------------------------

% \section{Methods}

% Maecenas sed ultricies felis. Sed imperdiet dictum arcu a egestas.
% \begin{itemize}
%   \item Donec dolor arcu, rutrum id molestie in, viverra sed diam
%   \item Curabitur feugiat
%   \item turpis sed auctor facilisis
%   \item arcu eros accumsan lorem, at posuere mi diam sit amet tortor
%   \item Fusce fermentum, mi sit amet euismod rutrum
%   \item sem lorem molestie diam, iaculis aliquet sapien tortor non nisi
%   \item Pellentesque bibendum pretium aliquet
% \end{itemize}
% \blindtext % Dummy text

% Text requiring further explanation\footnote{Example footnote}.

% %------------------------------------------------

% \section{Results}

% \begin{table}
%   \caption{Example table}
%   \centering
%   \begin{tabular}{llr}
%     \toprule
%     \multicolumn{2}{c}{Name}       \\
%     \cmidrule(r){1-2}
%     First name & Last Name & Grade \\
%     \midrule
%     John       & Doe       & $7.5$ \\
%     Richard    & Miles     & $2$   \\
%     \bottomrule
%   \end{tabular}
% \end{table}

% \blindtext % Dummy text

% \begin{equation}
%   \label{eq:emc}
%   e = mc^2
% \end{equation}

% \blindtext % Dummy text

% %------------------------------------------------

% \section{Discussion}

% \subsection{Subsection One}

% A statement requiring citation \cite{Figueredo:2009dg}. \blindtext % Dummy text

% \subsection{Subsection Two}

% \blindtext % Dummy text

%----------------------------------------------------------------------------------------
%	REFERENCE LIST
%----------------------------------------------------------------------------------------

% \begin{thebibliography}{99} % Bibliography - this is intentionally simple in this template

% \bibitem[Figueredo and Wolf, 2009]{Figueredo:2009dg}
% Figueredo, A.~J. and Wolf, P. S.~A. (2009).
% \newblock Assortative pairing and life history strategy - a cross-cultural
%   study.
% \newblock {\em Human Nature}, 20:317--330.

% \end{thebibliography}

\bibliographystyle{abbrv}
\bibliography{paper/bibliography}

%----------------------------------------------------------------------------------------

\end{document}
